%This is my super simple Real Analysis Homework template

\documentclass{article}
\usepackage[utf8]{inputenc}
\usepackage[english]{babel}
\usepackage[]{amsthm} %lets us use \begin{proof}
\usepackage[]{amssymb} %gives us the character \varnothing

\title{Assignment 3}
\author{Brett Sumser}
\date\today
%This information doesn't actually show up on your document unless you use the maketitle command below

\begin{document}
\maketitle %This command prints the title based on information entered above


\section*{Problem 1}

$ \begin{bmatrix}
0   & inf & inf & inf & -1  & inf \\
1   & 0   & inf & 2   & 0   & inf \\
inf & 2   & 0   & inf & inf & -8  \\
-4  & inf & inf & 0   & 3   & inf \\
inf & 7   & inf & inf & 0   & inf \\
inf & 5   & 10  & inf & inf & 0
\end{bmatrix}  $


$ \begin{bmatrix}
0   & inf & inf & inf & -1  & inf       \\
1   & 0   & inf & 2   & 0   & inf \\
inf & 2   & 0   & inf & inf & -8  \\
-4  & inf & inf & 0   & -5  & inf \\
inf & 7   & inf & inf & 0   & inf \\
inf & 5   & 10  & inf & inf & 0
\end{bmatrix}  $

$ \begin{bmatrix}
0   & inf & inf & inf & -1  & inf       \\
1   & 0   & inf & 2   & 0   & inf \\
inf & 2   & 0   & inf & inf & -8  \\
-4  & inf & inf & 0   & -5  & inf \\
inf & 7   & inf & inf & 0   & inf \\
inf & 5   & 10  & inf & inf & 0
\end{bmatrix}  $

$ \begin{bmatrix}
0   & inf & inf & inf & -1  & inf       \\
1   & 0   & inf & 2   & 0   & inf \\
3   & 2   & 0   & 4   & inf & -8  \\
-4  & inf & inf & 0   & -5  & inf \\
inf & 7   & inf & inf & 0   & inf \\
inf & 5   & 10  & inf & inf & 0
\end{bmatrix}  $

$ \begin{bmatrix}
0   & inf & inf & inf & -1  & inf       \\
1   & 0   & inf & 2   & 0   & inf \\
3   & 2   & 0   & 4   & 2   & -8  \\
-4  & inf & inf & 0   & -5  & inf \\
inf & 7   & inf & inf & 0   & inf \\
inf & 5   & 10  & inf & inf & 0
\end{bmatrix}  $

$ \begin{bmatrix}
0   & inf & inf & inf & -1  & inf       \\
1   & 0   & inf & 2   & 0   & inf \\
3   & 2   & 0   & 4   & 2   & -8  \\
-4  & inf & inf & 0   & -5  & inf \\
8   & 7   & inf & inf & 0   & inf \\
inf & 5   & 10  & inf & inf & 0
\end{bmatrix}  $

$ \begin{bmatrix}
0   & inf & inf & inf & -1  & inf       \\
1   & 0   & inf & 2   & 0   & inf \\
3   & 2   & 0   & 4   & 2   & -8  \\
-4  & inf & inf & 0   & -5  & inf \\
8   & 7   & inf & 9   & 0   & inf \\
inf & 5   & 10  & inf & inf & 0
\end{bmatrix}  $

$ \begin{bmatrix}
0   & inf & inf & inf & -1  & inf       \\
1   & 0   & inf & 2   & 0   & inf \\
3   & 2   & 0   & 4   & 2   & -8  \\
-4  & inf & inf & 0   & -5  & inf \\
8   & 7   & inf & 9   & 0   & inf \\
6   & 5   & 10  & inf & inf & 0
\end{bmatrix}  $

$ \begin{bmatrix}
0   & inf & inf & inf & -1 & inf       \\
1   & 0   & inf & 2  & 0   & inf \\
3   & 2   & 0   & 4  & 2   & -8  \\
-4  & inf & inf & 0  & -5  & inf \\
8   & 7   & inf & 9  & 0   & inf \\
6   & 5   & 10  & 7  & inf & 0
\end{bmatrix}  $


$ \begin{bmatrix}
0   & inf & inf & inf & -1 & inf      \\
1   & 0   & inf & 2  & 0  & inf \\
3   & 2   & 0   & 4  & 2  & -8  \\
-4  & inf & inf & 0  & -5 & inf \\
8   & 7   & inf & 9  & 0  & inf \\
6   & 5   & 10  & 7  & 5  & 0
\end{bmatrix}  $


$ \begin{bmatrix}
0   & inf & inf & inf & -1 & inf      \\
-2  & 0   & inf & 2  & 0  & inf \\
3   & 2   & 0   & 4  & 2  & -8  \\
-4  & inf & inf & 0  & -5 & inf \\
8   & 7   & inf & 9  & 0  & inf \\
6   & 5   & 10  & 7  & 5  & 0
\end{bmatrix}  $


$ \begin{bmatrix}
0   & inf & inf & inf & -1 & inf      \\
-2  & 0   & inf & 2  & -3 & inf \\
3   & 2   & 0   & 4  & 2  & -8  \\
-4  & inf & inf & 0  & -5 & inf \\
8   & 7   & inf & 9  & 0  & inf \\
6   & 5   & 10  & 7  & 5  & 0
\end{bmatrix}  $

$ \begin{bmatrix}
0   & inf & inf & inf & -1 & inf      \\
-2  & 0   & inf & 2  & -3 & inf \\
0   & 2   & 0   & 4  & 2  & -8  \\
-4  & inf & inf & 0  & -5 & inf \\
8   & 7   & inf & 9  & 0  & inf \\
6   & 5   & 10  & 7  & 5  & 0
\end{bmatrix}  $

$ \begin{bmatrix}
0   & inf & inf & inf & -1 & inf      \\
-2  & 0   & inf & 2  & -3 & inf \\
0   & 2   & 0   & 4  & -1 & -8  \\
-4  & inf & inf & 0  & -5 & inf \\
8   & 7   & inf & 9  & 0  & inf \\
6   & 5   & 10  & 7  & 5  & 0
\end{bmatrix}  $

$ \begin{bmatrix}
0   & inf & inf & inf & -1 & inf      \\
-2  & 0   & inf & 2  & -3 & inf \\
0   & 2   & 0   & 4  & -1 & -8  \\
-4  & inf & inf & 0  & -5 & inf \\
8   & 7   & inf & 9  & 0  & inf \\
6   & 5   & 10  & 7  & 5  & 0
\end{bmatrix}  $


$ \begin{bmatrix}
0   & inf & inf & inf & -1 & inf      \\
-2  & 0   & inf & 2  & -3 & inf \\
0   & 2   & 0   & 4  & -1 & -8  \\
-4  & inf & inf & 0  & -5 & inf \\
5   & 7   & inf & 9  & 0  & inf \\
3   & 5   & 10  & 7  & 5  & 0
\end{bmatrix}  $


$ \begin{bmatrix}
0   & inf & inf & inf & -1 & inf      \\
-2  & 0   & inf & 2  & -3 & inf \\
0   & 2   & 0   & 4  & -1 & -8  \\
-4  & inf & inf & 0  & -5 & inf \\
5   & 7   & inf & 9  & 0  & inf \\
3   & 5   & 10  & 7  & 2  & 0
\end{bmatrix}  $


$ \begin{bmatrix}
0   & 6 & inf & inf & -1 & inf      \\
-2  & 0   & inf & 2  & -3 & inf \\
0   & 2   & 0   & 4  & -1 & -8  \\
-4  & inf & inf & 0  & -5 & inf \\
5   & 7   & inf & 9  & 0  & inf \\
3   & 5   & 10  & 7  & 2  & 0
\end{bmatrix}  $

$ \begin{bmatrix}
0   & 6 & inf & inf & -1 & inf      \\
-2  & 0   & inf & 2  & -3 & inf \\
0   & 2   & 0   & 4  & -1 & -8  \\
-4  & inf & inf & 0  & -5 & inf \\
5   & 7   & inf & 9  & 0  & inf \\
3   & 5   & 10  & 7  & 2  & 0
\end{bmatrix}  $


$ \begin{bmatrix}
0 & 6 & inf & 8   & -1 & inf      \\
-2    & 0   & inf & 2  & -3 & inf \\
0     & 2   & 0   & 4  & -1 & -8  \\
-4    & 2   & inf & 0  & -5 & inf \\
5     & 7   & inf & 9  & 0  & inf \\
3     & 5   & 10  & 7  & 2  & 0
\end{bmatrix}  $


$ \begin{bmatrix}
0 & 6 & inf & 8   & -1 & inf      \\
-2    & 0   & inf & 2  & -3 & inf \\
0     & 2   & 0   & 4  & -1 & -8  \\
-4    & 2   & inf & 0  & -5 & inf \\
5     & 7   & inf & 9  & 0  & inf \\
3     & 5   & 10  & 7  & 2  & 0
\end{bmatrix}  $


$ \begin{bmatrix}
0 & 6 & inf & 8 & -1 & inf \\
-2 & 0 & inf & 2 & -3 & inf \\
0 & 2 & 0 & 4 & -1 & -8 \\
-4 & 2 & inf & 0 & -5 & inf \\
5 & 7 & inf & 9 & 0 & inf \\
3 & 5 & 10 & 7 & 2 & 0
\end{bmatrix}  $


$ \begin{bmatrix}
0 & 6 & inf & 8 & -1 & inf \\
-2 & 0 & inf & 2 & -3 & inf \\
-5 & -3 & 0 & -1 & -1 & -8 \\
-4 & 2 & inf & 0 & -5 & inf \\
5 & 7 & inf & 9 & 0 & inf \\
3 & 5 & 10 & 7 & 2 & 0
\end{bmatrix}  $


$ \begin{bmatrix}
0 & 6 & inf & 8 & -1 & inf \\
-2 & 0 & inf & 2 & -3 & inf \\
-5 & -3 & 0 & -1 & -1 & -8 \\
-4 & 2 & inf & 0 & -5 & inf \\
5 & 7 & inf & 9 & 0 & inf \\
3 & 5 & 10 & 7 & 2 & 0
\end{bmatrix}  $


\section*{Problem 2}

For this problem, you need to find the greatest distance that can be travelled between two towns with only a single pit stop.

Expressing that as $c[i,j,k]$, (max distance between cities i and j stopping in city k for fuel), you can describe a recurrence relation by:

$c[i,j,k] = max(c[i,j,x] + c[k,j,x])$ for each x in the set of vertices ($V$). Therefore the expression for single stop capacity would be:

$max{c[i,j,k] | I_j,k \in V}$

\section*{Problem 3}

Start by assuming that a sign will be placed at the i'th milepost. To calculate the next best milestone, you must
try every next milepost and reccur until reaching the last milepost. So the cost for placing a sign at milepost $m_i$, would be:

$cost(m_i) = min(cost(m_j)) + cost at m_j \forall j > i$

Signs will also be placed at $m_1=0$ and $m_n$.

The cost for placing a sign at $m_j$ is $(100 - (j-i))^2$. At each $m_1...m_n$, you need to seperatly calculate the costs and add in the minimum.
If you store the calculated costs in a table, instead of recalculating them, you can reduce the number of computations quite a bit.

The reccurence would be:

$cost(i) = (100 - (j-i))^2 + min(cost(j)) \forall j>i \and j<=n$

\section*{Problem 4}

For each $i$th day, it depends on the cost of the $i-1$th day, and so on. By mainting a dynamic programmingh table, you can store the minimum
operating costs of both cities $C$ and $D$. To decide on moving for each day, the value would have to be $M + (min) < max$ value.

Given the operating cost $O(i, c)$, such that for all days up to day $i$ and ending up in city $c$ ($c_1 = start$, $c_2 = end$)
and moving cost $M(c_1, c_2)$, the recurence would be:

$O(i, c) = min(O(i-1, c_1), min(O(i-1, c_2) + M(c_1, c_2) \forall $ cities such that $ c_2 != c_1)) + A[c][i]$

The recurence for the optimal value can be determined by maintaining a minimum array of both cities operating costs.


\end{document}
